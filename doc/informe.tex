\documentclass[11pt,a4paper, spanish]{article}
\usepackage[T1]{fontenc}
\usepackage[utf8]{inputenc}
\usepackage[spanish]{babel}
\selectlanguage{spanish}
\usepackage{graphicx}
\usepackage{listings}

\lstset{
    language=C,
    tabsize=4,
    basicstyle=\fontsize{11}{13}\ttfamily\footnotesize,
    showspaces=false,
    showstringspaces=false,
    captionpos=b,
    breaklines=true,
    literate={á}{{\'a}}1
        {ã}{{\~a}}1
        {é}{{\'e}}1
        {ó}{{\'o}}1
        {í}{{\'i}}1
        {ñ}{{\~n}}1
        {¡}{{!`}}1
        {¿}{{?`}}1
        {ú}{{\'u}}1
        {Í}{{\'I}}1
    {Ó}{{\'O}}1
}

\usepackage{multirow}
\usepackage{float}
\usepackage[caption = false]{subfig}

\setcounter{secnumdepth}{0}

\begin{document}

%  FRONTPAGE

\begin{titlepage}
  \noindent%
  \begin{tabular}{@{}p{\textwidth}@{}}
    \vspace{0.2cm}
    \begin{center}
    \Huge{\textbf{
      Data Path
    }}
    \end{center}
    \begin{center}
      \Large{
         66:20 Organizaci\'on de Computadoras
      }
    \end{center}
    \vspace{0.2cm}\\
  \end{tabular}
  \vspace{4 cm}
  \begin{center}
    {\large
      Trabajo práctico 3
    }\\
    \vspace{0.6cm}
    {\Large
    Mauro Toscano (96890)\\
    Axel Lijdens (95772)
    }
  \end{center}
  \vfill
  \begin{center}
  Univesidad de Buenos Aires - FIUBA
  \end{center}
\end{titlepage}


\tableofcontents
\pagebreak
% Prologo

\section{Objetivos}

El objetivo de este trabajo es familiarizarse con la arquitectura de una CPU MIPS, espec\'ficamente con el datapath y la implementación de instrucciones. Para ello, se deberían agregar instrucciones a diversas configuraciones de CPU provistas por el simulador DrMIPS

\section{Alcance}

Este trabajo práctico es de elaboración grupal, evaluación individual, y de carácter obligatorio para todos alumnos del curso.

\section{Requisitos}

El trabajo deberá ser entregado personalmente, en la fecha estipulada, con una carátula que contenga los datos completos de todos los integrantes.

Además, es necesario que el trabajo práctico incluya (entre otras cosas, ver sección 8), la presentación de los resultados obtenidos, explicando, cuando corresponda, con fundamentos reales, las causas o razones de cada resultado obtenido. Por este motivo, el día de la entrega deben concurrir todos los integrantes del grupo.

El informe deber\'a respetar el modelo de referencia que se encuentra en el grupo, y se valorar\'an aquellos escritos usando la herramienta TEX / LATEX.

\section{Recursos}

Usaremos el programa DrMIPS para configurar y simular el data path de un procesador MIPS, tanto uniciclo como multiciclo.

\section{Fecha de entrega}

La última fecha de entrega y presentación ser\'a el jueves 28 de junio de 2018.

\section{Introducción}\label{informe}
El programa DrMIPS nos permite evaluar distintos diseños de datapath para procesadores
MIPS32, al darnos la posibilidad de organizarlo como queramos. Si bien sólo puede haber uno
de algunos de los componentes del DP (como el registro de PC o la unidad de control), podemos
poner sumadores, multiplexores, extensores de signo y conexiones arbitrariamente. También es
posible modificar el conjunto de instrucciones. Además de la estructura lógica del DP, DrMips
nos permite escribir programas simples y simular su ejecución en el DP, mostrando los valores
que toman las diversas entradas y salidas de cada elemento. El programa se puede conseguir en
https://bitbucket.org/brunonova/drmips/wiki/Home, o se puede descargar para Ubuntu,
ya sea desde el repositorio de Ubuntu (aunque la versión está desactualizada) o autorizando un
repositorio externo.

% Desarrollo

\section{Instrucciones a implementar}

\begin{enumerate}
  \item Implementar la instrucción \texttt{j} en el DP \texttt{pipeline.cpu}.
  \item Implementar la instrucción \texttt{jr} (Jump Register) en el DP \texttt{unicycle.cpu}.
  \item Implementar la instrucción \texttt{jr} en el DP \texttt{pipeline.cpu}. Verificar que no se produzcan hazards.
  \item Implementar la instrucción \texttt{jalr} (Jump and Link Register) en el DP \texttt{unicycle.cpu}.
  \item Implementar la instrucción \texttt{jalr} en el DP \texttt{pipeline.cpu}. Verificar que no se produzcan hazards.
\end{enumerate}


\section{Pruebas}

En todos los casos debe verificarse que la instrucción se ejecute correctamente. Esto implica
que el PC tome el valor deseado, y además que en el caso del DP multiciclo no se produzcan
hazards, como ser la ejecución de la instrucción siguiente al salto, o en el caso de utilizar el valor
de un registro, que éste tenga el valor correcto.


\section{Conclusiones}


% SOlO codigos fuente hacia abajo, no hace falta editar!

\newpage

\section{Archivos .cpu}

Repositorio: https://github.com/MauroFab/orga-6620-tp3


\end{document}
\grid
